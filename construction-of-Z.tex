\documentclass[12pt]{article}

\usepackage{amsmath}
\usepackage{relsize}
\usepackage[margin=1in, paperwidth=8.5in, paperheight=13in]{geometry}
\usepackage{amsthm}
\usepackage{amsfonts}
\usepackage{amssymb}
\usepackage{parskip}
\usepackage{microtype}

\pagenumbering{gobble}
\renewcommand{\thesection}{\Roman{section}} 
\renewcommand{\thesubsection}{\thesection.\Roman{subsection}}
\newcommand{\stext}[1]{\quad\text{\small #1}}
\theoremstyle{definition}

\newtheorem{theorem}{Theorem}
\newtheorem*{claim}{Claim}
\newtheorem{definition}{Definition}
\newtheorem*{axiom}{Axiom}
\newtheorem{lemma}{Lemma}
\newtheorem*{remark}{Remark}
\newtheorem*{example}{Example}
\allowdisplaybreaks

\begin{document}
\begin{center}
    \large \textbf{inventing numbers and number construction} \\
    {part 2 the integers and stuff} \\

    {\small it is now very sunny thankfully\\}
\end{center}

\begin{definition}[Set of Integers] We will construct a set $\mathbb{Z}$ with its elements that is denoted by the following equivalence class: for arbritrary $p, q, r, s \in \mathbb{N}$, two integers denoted by their equivalence classes $(p, q)\sim (r, s)$ if and only if \(p + s = r + q\). Let $k, t \in \mathbb{N}$. An equivalent definition, and is the way to construct the elements in relation to each other, is shown below.
    \begin{align*}
        0 &:= [(k, k)] \\ 
        1 &:= [(k + 1, k)] \\
        -1 &:= [(k, k + 1)] \\
        2 &:= [(k + 2, k)] \\
        -2 &:= [(k, k + 2)] \\
        3 &:= [(k + 3, k)] \\
        ... \\
        z &:=  [(k + t, k)] \\ 
        -z &:=  [(k, k + t)]
    \end{align*}
\end{definition}

\begin{theorem}
    The set $\mathbb{Z}$ has infinite elements. 
\end{theorem}
\begin{proof}
    By contradiction, suppose that $\mathbb{Z}$ has finite elements. Define $a_n \subseteq \mathbb{Z}$ to be a sequence of integers generated by an equivalence class such that 
    \begin{equation*}
        a_n = \{[(k + n, k)]\}
    \end{equation*}
    Since $\mathbb{Z}$ is finite, all of its subsets must also be finite. By the Fundamental Axiom, $\mathbb{N}$ is infinite. Since $\mathbb{N}$ is infinite and we can always take an $S(n) \in \mathbb{N}$ to be greater than the previous $n$, this implies that we can always generate a next integer in $a_n$ for all $n \in \mathbb{N}$. This implies that $a_n$ is infinite. Since $a_n$ is a subset of $\mathbb{Z}$, this is a contradiction to our assumption that $\mathbb{Z}$ is finite. Thus, $\mathbb{Z}$ must be infinite.
\end{proof}

\begin{definition}[Addition on $\mathbb{Z}$] 
    For all $a, b \in \mathbb{Z}$, we will define addition as a function $+: \mathbb{Z}^2 \to \mathbb{Z}$ given by the following mapping relation
    \begin{align*}
        \intertext{Writing $a$ and $b$ by their equivalence classes}
        a &= [(p, q)] \\
        b &= [(r, s)] \\ 
        a + b &:= [(p + r, q + s)]
    \end{align*}
\end{definition}

\begin{theorem}[Identity Element of Addition on $\mathbb{Z}$]
    \begin{equation*}
        a + 0 = a = 0 + a
    \end{equation*}
\end{theorem}
\begin{proof}
    \begin{align*}
        a &= [(p, q)] \\ 
        0_{\mathbb{Z}} &= [(k, k)]
    \end{align*}
    Let $k = 0$ be the generator of the representatitive of the equivalence class of $0$.
    \begin{align*}
        & a + 0 \\
        &= [(p, q)] + [(0, 0)] \stext{by Substitution}\\
        &=[(p + 0, q + 0)] \stext{by Addition on $\mathbb{Z}$} \\ 
        &= [(p, q)] = a \stext{by Identity Definition of Addition on $\mathbb{N}$} \\ \\
        & 0 + a \\ 
        &= [(0, 0)] + [(p, q)] \stext{by Substitution}\\
        &=[(0 + p, 0 + q)] \stext{by Addition in $\mathbb{Z}$} \\ 
        &= [(p, q)] = a \stext{by Left Identity of Addition on $\mathbb{N}$}
    \end{align*}
\end{proof}

\begin{definition}[Additive Inverse on $\mathbb{Z}$]
    For all $a \in \mathbb{Z}$ we define the additive inverse of $a$ to be $b$ if and only if $a + b = 0$
\end{definition}
\begin{theorem}
    For all $a \in \mathbb{Z}$, the additive inverse of $a$ is given by $-a$
    \begin{align*}
        a &= [(p, q)] \\ 
        -a &= [(q, p)]
    \end{align*}
\end{theorem}
\begin{proof}
    \begin{align*}
        & a + (-a) \\
        & [(p, q)] + [(q, p)] \stext{by Substitution}\\ 
        & [(p + q, q + p)] \stext{by Definition of Addition on $\mathbb{Z}$}\\ 
        & [(p + q, p + q)] \stext{by Commutativity}\\ 
        \intertext{By the closure of addition on $\mathbb{N}$, let $k \in \mathbb{N} = p + q$ and substituting}
        & [(k, k)] = 0 \stext{by Definition of 0 on $\mathbb{Z}$}
    \end{align*}
\end{proof}
\begin{remark}
    It follows from this that $0$ is its own additive inverse.
\end{remark}

\begin{theorem}[Associativity of Addition on $\mathbb{Z}$]
    For all $a, b, c \in \mathbb{Z}$
    \begin{equation*}
        (a + b) + c = a + (b + c)
    \end{equation*}
\end{theorem}
\begin{proof}
    \begin{align*}
        a &= [(p, q)] \\ 
        b &= [(r, s)] \\ 
        c &= [(t, u)] 
    \end{align*}
    \begin{align*}
        & (a + b) + c \\
        & ([(p, q)] + [(r, s)]) + [(t, u)] \stext{by Substitution}\\
        & [(p + r, q + s)] + [(t, u)] \stext{by Definition of Addition on $\mathbb{Z}$}\\ 
        & [((p + r) + t, (q + s) + u)] \stext{by Definition of Addition on $\mathbb{Z}$}\\
        & [(p + (r + t), q + (s + u))] \stext{by Associativity of Addition on $\mathbb{Z}$}\\
        & [(p + q)] + [(r + t, s + u)] \stext{by Definition of Addition on $\mathbb{Z}$}\\ 
        & [(p + q)] + ([(r, s)] + [(t, u)]) \stext{by Definition of Addition on $\mathbb{Z}$}\\ 
        & a + (b + c) \stext{by Definition of Addition on $\mathbb{Z}$}
    \end{align*}
\end{proof}

\begin{theorem}[Commutativity of Addition on $\mathbb{Z}$]
    For all $a, b \in \mathbb{Z}$
    \begin{equation*}
        a + b = b + a
    \end{equation*}
\end{theorem}
\begin{proof}
    \begin{align*}
        a &= [(p, q)] \\ 
        b &= [(r, s)]
    \end{align*}
    \begin{align*}
        & a + b \\ 
        & [(p, q)] + [(r, s)] \stext{by Substitution}\\
        & [(p + r), (q + s)] \stext{by Definition of Addition on $\mathbb{Z}$}\\ 
        & [(r + p), (s + q)] \stext{by Commutativity of Addition on $\mathbb{N}$}\\ 
        & [(r + s)] + [(p, q)] \stext{by Definition of Addition on $\mathbb{Z}$}\\ 
        & b + a \stext{by Substitution}
    \end{align*}
\end{proof}

\begin{definition}[Partial Order $\geq$ on $\mathbb{Z}$]
    \begin{align*}
        a &= [(p, q)] \\
        b &= [(r, s)] \\
    \end{align*}
    \begin{equation*}
        a \geq b \iff p + s \geq q + r
    \end{equation*}
\end{definition}

\begin{theorem}[Reflexivity of $\geq$ on $\mathbb{Z}$]
    For all $a \in \mathbb{Z}$
    \begin{equation}
        a \geq a
    \end{equation}
\end{theorem}

\begin{proof}
    \begin{equation*}
        a = [(p, q)] 
    \end{equation*}
    \begin{align*}
        &p + q \geq p + q \stext{by Reflexivity of $\geq$ on $\mathbb{N}$} \\
        & a \geq a \stext{by Definition of $\geq$ on $\mathbb{Z}$}
    \end{align*}
\end{proof}
\begin{theorem}[Transitivity of $\geq$ on $\mathbb{Z}$]
    For all $a, b, c \in \mathbb{Z}$ if $a \geq b$ and $b \geq$ c then $a \geq c$
\end{theorem}
\begin{proof}
    \begin{align*}
        a &= [(p, q)] \\ 
        b &= [(r, s)] \\ 
        c &= [(t, u)]
    \end{align*}
    \begin{align*}
        p + s &\geq q + r \stext{by Definition of $\geq$ on $\mathbb{Z}$}\\
        r + u &\geq s + t \stext{by Definition of $\geq$ on $\mathbb{Z}$}\\ 
        (p + s) + (r + u) &\geq (q + r) + (s + t) \stext{Combining Inequalities}\\
        (s + p) + (u + r) &\geq (r + q) + (t + s) \stext{by Commutativity of Addition on $\mathbb{N}$}\\ 
        s + (p + u) + r &\geq r + (q + t) + s \stext{by Associativity of Addition on $\mathbb{N}$}\\
        s + r + (p + u) &\geq r + s + (q + t) \stext{by Commutativity of Addition on $\mathbb{N}$}\\
        s + r + (p + u) &\geq s + r + (q + t) \stext{by Commutativity of Addition on $\mathbb{N}$}\\
        p + u &\geq q + t \stext{Applying Inverse Addition on Both Sides}\\ 
        [(p + q)] &\geq [(t, u)] \stext{by Definition of Addition on $\mathbb{Z}$}\\
        a &\geq c \stext{by Substitution}
    \end{align*}
\end{proof}
\begin{theorem}[Antisymmetry of $\geq$ on $\mathbb{Z}$]
    For all $a, b \in \mathbb{Z}$ if $a \geq b$ and $b \geq a$ then $a = b$
\end{theorem}
\begin{proof}
    \begin{align*}
        a &= [(p, q)] \\ 
        b &= [(r, s)]
    \end{align*}
    \begin{align*}
        & p + s \geq q + r \\
        & r + q \geq s + p \implies q + r \geq p + s \stext{by Commutativity of Addition on $\mathbb{N}$}\\ 
        & p + s = q + r \stext{by Antisymmetry (Line 1, 2) of Addition on $\mathbb{N}$ }\\
        & [(p, q)] = [(r, s)] \stext{by Definition of Equivalence Class on $\mathbb{Z}$}\\ 
        & a = b \stext{by Substitution}
    \end{align*}
\end{proof}

\begin{definition}[Multiplication on $\mathbb{Z}$]
    \begin{align*}
        a &= [(p, q)] \\ 
        b &= [(r, s)]
    \end{align*}
    \begin{align*}
        a \cdot b &:= [(p, q)] \cdot [(r, s)] \\
        &= [((p \cdot r) + (q \cdot s), (p \cdot s) + (q \cdot r))]
    \end{align*}
\end{definition}
\begin{theorem}[Identity Element of Multiplication on $\mathbb{Z}$]
    For all $a \in \mathbb{Z}$
    \begin{equation*}
        a \cdot 1 = 1 \cdot a = a
    \end{equation*}
\end{theorem}
\begin{proof}
    \begin{align*}
        a &= [(p, q)] \\
        1 &= [(k + 1, k)]
    \end{align*}
    \begin{align*}
        \intertext{Let $k = 0$ be the generator of the representatitive of the equivalence class of $1$.}
        & a \cdot [(0 + 1, 0)] \stext{by Substitution}\\ 
        & a \cdot [(1, 0)] \stext{by Left Identity of Addition on $\mathbb{N}$}\\ 
        & [(p, q)] \cdot [(1, 0)] \stext{by Substitution}\\
        & [((p \cdot 1) + (q \cdot 0), (p \cdot 0) + (q \cdot 1))] \stext{by Definition of Multiplication on $\mathbb{Z}$}\\ 
        & [((p \cdot 1) + 0, 0 + (q \cdot 1))] \stext{by L/R Annihilator on $\mathbb{N}$}\\ 
        & [(p \cdot 1, q \cdot 1)] \stext{by L/R Identity of Addition on $\mathbb{N}$}\\ 
        & [(p, q)] \stext{by Identity of Multiplication on $\mathbb{N}$}\\
        & a \stext{by Substitution}\\
    \end{align*}
\end{proof}
\begin{theorem}[Annihilator of Multiplication on $\mathbb{Z}$]
    For all $a \in \mathbb{Z}$
    \begin{equation*}
        a \cdot 0 = 0 \cdot a = 0
    \end{equation*}
\end{theorem}
\begin{proof}
    \begin{align*}
        a &= [(p, q)] \\ 
        0 &= [(k, k)]
    \end{align*}
    \begin{align*}
        \intertext{Let $k = 0$ be the generator of the representatitive of the equivalence class of $0$.}
        & a \cdot 0 \\ 
        & [(p, q)] \cdot [(0, 0)] \stext{by Substitution}\\ 
        & [(p \cdot 0) + (q \cdot 0), (p \cdot 0), (q \cdot 0)] \stext{by Definition of Multiplication on $\mathbb{Z}$}\\ 
        & [(0, 0)] \stext{by L/R Annihilator on $\mathbb{N}$}\\
        \sim & [(k, k)] = 0 \stext{by Definition of $0$ on $\mathbb{Z}$}\\ \\ 
        & 0 \cdot a \\
        & [(0, 0)] \cdot [(p, q)] \stext{by Substitution}\\ 
        & [(0 \cdot p) + (0 \cdot q), (0 \cdot q), (0 \cdot p)] \stext{by Definition of Multiplication on $\mathbb{Z}$}\\ 
        & [(0, 0)] \stext{by L/R Annihilator on $\mathbb{N}$} \\
        \sim & [(k, k)] = 0 \stext{by Definition of $0$ on $\mathbb{Z}$}
    \end{align*}
\end{proof}
\begin{theorem}[Associativity of Multiplication on $\mathbb{Z}$]
    For all $a, b, c \in \mathbb{Z}$
    \begin{equation*}
        (a \cdot b) \cdot c = a \cdot (b \cdot c)
    \end{equation*}
\end{theorem}
\begin{proof}
    \begin{align*}
        a &= [(p, q)] \\ 
        b &= [(r, s)] \\ 
        c &= [(t, u)]
    \end{align*}
    \begin{align*}
        & (a \cdot b) \cdot c \\
        & ([(p, q)] \cdot [(r, s)]) \cdot [(t, u)] \stext{by Definition}\\ 
        & [((p \cdot r) + (q \cdot s), (p \cdot s) + (q \cdot r))] \cdot [(t, u)] \stext{by Definition of Multiplication on $\mathbb{Z}$} 
        \intertext{by Definition of Multiplication on $\mathbb{Z}$:}
        & [((((p \cdot r) + (q \cdot s)) \cdot t) + ((p \cdot s) + (q \cdot r) \cdot u), (((p \cdot r) + (q \cdot s)) \cdot u) + ((p \cdot s) + (q \cdot r)) \cdot t)]
        \intertext{by Associativity of Multiplication on $\mathbb{N}$:}
        & [((((p \cdot r) + q \cdot (s \cdot t))) + ((p \cdot s) + q \cdot (r \cdot u)), (((p \cdot r) + q \cdot (s \cdot u))) + ((p \cdot s) + (q \cdot (r \cdot t))))] 
        \intertext{by L/R Distributivity of Multiplication over Addition on $\mathbb{N}$:}
        & [(p \cdot ((r \cdot t) + (s \cdot u))) + (q \cdot ((r \cdot u) + (s \cdot t))), (p \cdot ((r \cdot u) + (s \cdot t))) + (q \cdot ((r \cdot t) + (s \cdot u)))] 
        \intertext{by Definition of Multiplication on $\mathbb{Z}$:}
        & [(p, q)] \cdot [((r \cdot t) + (s \cdot u), (r \cdot u) + (s \cdot t))]
        \intertext{by Definition of Multiplication on $\mathbb{Z}$:}
        & [(p, q)] \cdot ([(r, s)] \cdot [(t, u)]) 
        \intertext{by Substitution}
        & a \cdot (b \cdot c)
    \end{align*}
\end{proof}
\begin{theorem}[Commutativity of Multiplication on $\mathbb{Z}$]
    For all $a, b \in \mathbb{Z}$
    \begin{equation*}
        a \cdot b = b \cdot a
    \end{equation*}
\end{theorem}
\begin{proof}
    \begin{align*}
        a &= [(p, q)] \\ 
        b &= [(r, s)] \\ 
    \end{align*}
    \begin{align*}
        & a \cdot b \\
        & [(p, q)] \cdot [(r, s)] \stext{by Substitution}\\ 
        & [((p \cdot r) + (q \cdot s), (p \cdot s) + (q \cdot r))] \stext{by Definition of Multiplication on $\mathbb{Z}$}\\ 
        & [((r \cdot p) + (s \cdot q), (s \cdot p) + (r \cdot q))] \stext{by Commutativity of Multiplication on $\mathbb{N}$}\\ 
        & [(r, s)] \cdot [(q, p)] \stext{by Definition of Multiplication on $\mathbb{Z}$}\\ 
        & b \cdot a \stext{by Substitution}
    \end{align*}
\end{proof}
\begin{theorem}[Distributivity of Multiplication over Addition on $\mathbb{Z}$]
    \begin{align*}
        (a + b) \cdot c &= a \cdot c + b \cdot c \\ 
        c \cdot (a + b) &= c \cdot a + c \cdot b
    \end{align*}
\end{theorem}
\begin{proof}
    \begin{align*}
        a &= [(p, q)] \\ 
        b &= [(r, s)] \\ 
        c &= [(t, u)]
    \end{align*}
    \begin{align*}
        & (a + b) \cdot c \\
        & ([(p, q)] + [(r, s)]) \cdot [(t, u)] \stext{by Substitution} \\
        & [(p + r, q + s)] \cdot [(t, u)] \stext{by Definition of Addition on $\mathbb{N}$}
        \intertext{by Definition of Multiplication on $\mathbb{N}$:}
        & [((p + r) \cdot t + (q + s) \cdot u, (p + r) \cdot u + (q + s) \cdot t)]
        \intertext{by L/R Distributivity of Multiplication over Addition on $\mathbb{N}$:}
        & [((p \cdot t + r \cdot t) + (q \cdot u + s \cdot u), (p \cdot u + r \cdot u) + (q \cdot t + s \cdot t))]
        \intertext{by Associativity of Addition on $\mathbb{N}$:}
        & [((p \cdot t) + (r \cdot t) + (q \cdot u) + (s \cdot u), (p \cdot u) + (r \cdot u) + (q \cdot t) + (s \cdot t))] 
        \intertext{by Commutativity of Addition on $\mathbb{N}$:}
        & [((p \cdot t) + (q \cdot u) + (r \cdot t) + (s \cdot u), (p \cdot u)  + (q \cdot t) + (r \cdot u) + (s \cdot t))] 
        \intertext{by Definition of Addition on $\mathbb{Z}$:}
        & [((p \cdot t) + (q \cdot u), (p \cdot u) + (q \cdot t))] + [((r \cdot t) + (s \cdot u), (r \cdot u) + (s \cdot t))] 
        \intertext{by Definition of Multiplication on $\mathbb{Z}$:}
        & [(p, q)] \cdot [(t, u)] + [(r, s)] \cdot [(t, u)] \\ 
        & a \cdot c + b \cdot c \stext{by Substitution}
    \end{align*}
    \begin{align*}
        (a + b) \cdot c &= a \cdot c + b \cdot c \stext{Proven above}\\
        c \cdot (a + b) &= a \cdot c + b \cdot c \stext{by Commutativity of Multiplication on $\mathbb{Z}$}\\
        &= c \cdot a + c \cdot b \stext{by Commutativity of Multiplication on $\mathbb{Z}$}
    \end{align*}
\end{proof}
\end{document}