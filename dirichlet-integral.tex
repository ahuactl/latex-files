\documentclass[12pt]{article}

\usepackage{amsmath}
\usepackage{relsize}
\usepackage[margin=1in, paperwidth=8.5in, paperheight=13in]{geometry}
\usepackage{amsthm}
\usepackage{amsfonts}
\usepackage{amssymb}
\usepackage{parskip}
\usepackage{microtype}

\pagenumbering{gobble}
\renewcommand{\thesection}{\Roman{section}} 
\renewcommand{\thesubsection}{\thesection.\Roman{subsection}}
\newcommand{\stext}[1]{\quad\text{\small #1}}
\theoremstyle{definition}

\newtheorem{theorem}{Theorem}
\newtheorem*{claim}{Claim}
\newtheorem*{definition}{Definition}
\newtheorem*{axiom}{Axiom}
\newtheorem{lemma}{Lemma}
\newtheorem*{remark}{Remark}
\newtheorem*{example}{Example}
\allowdisplaybreaks

\begin{document}
\begin{center}
    \large \textbf{dirichlet integral} \\
    {feynman was great some says} \\

    {\small the rain stopped and it's sunny\\}
\end{center}

The Dirichlet Integral for a non-zero real-valued $a$ is given by $I$, where
\begin{equation*}
    I = \int_{-\infty}^{\infty} \frac{\sin(ax)}{x} \,dx
\end{equation*}

\begin{lemma} The function $f$
    \begin{equation*}
        f(x) = \frac{\sin(ax)}{x}
    \end{equation*}
    is an even function.
\end{lemma}
\begin{proof}
    \begin{align*}
        f(-x) &=  \frac{\sin(-ax)}{-x} \\ 
        \intertext{Since $\sin t$ is an odd function, we can pop the negative sign out}
        &= \frac{-\sin(ax)}{-x} \\
        &= \frac{\sin(ax)}{x} = f(x) \\ 
        \implies f(-x) &= f(x)
    \end{align*}
\end{proof}

Since the integrand is an even function, by the properties of the integral, we can consider an integral from $0$ to $\infty$ and double the result. Motivated by Feynman's integration, consider a generalized function $g$ 
\begin{equation*}
    g(s, x) = e^{-sx} \frac{\sin(ax)}{x}
\end{equation*}
where our function $f$ is a special case when $s = 0$. We can then integrate $g$ from $0$ to $\infty$, take $s$ to be 0 and double the result. In that case, the value that is derived is equal to the original integral $I$. We will call the integral of $g$, $I(s)$.

\begin{definition}[Complex Exponential Definition of $\sin t$] For $t \in \mathbb{C}$
    \begin{equation*}
        \sin t = \frac{e^{it} - e^{-it}}{2i}
    \end{equation*}
\end{definition}

\begin{lemma}
    \begin{equation*}
        \lim_{s \to \infty} I(s) = 0
    \end{equation*}
\end{lemma}
\begin{proof}
    \begin{align*}
    \lim_{s \to \infty} I(s) &= \lim_{s \to \infty} \int_{0}^{\infty} e^{-sx} \frac{\sin(ax)}{x} \,dx \\ 
    \intertext{by the Dominated Convergence Theorem, the exchange of limit and integrals are allowed.}
    &=  \int_{0}^{\infty} \lim_{s \to \infty} e^{-sx} \frac{\sin(ax)}{x} \,dx \\
    \intertext{The whole integrand collapses to $0$ since the exponential decay term approaches 0 as $s \to \infty$. Thus, the whole integral is also $0$.}
    \lim_{s \to \infty} I(s) &= 0
    \end{align*}
\end{proof}

Using the tools from above, we can now evaluate our integral.
\begin{align*}
    % I &= \int_{-\infty}^{\infty} \frac{\sin(ax)}{x} \,dx \\
    I(s) &= \int_{0}^{\infty} e^{-sx} \frac{\sin(ax)}{x} \,dx \\
    \intertext{Taking the derivative of both sides}
    I'(s) &= \frac{d}{ds} \int_{0}^{\infty} e^{-sx} \frac{\sin(ax)}{x} \,dx \\
    \intertext{Since the integrand is bounded, we can interchange the derivative and integral operators. By Leibniz' Integral Rule, the total derivative turns into a partial one inside the integral.}
    &= \int_{0}^{\infty} \frac{\partial}{\partial s} \left(e^{-sx} \frac{\sin(ax)}{x}\right) \,dx \\
    \intertext{Taking the partial derivative with respect to $s$}
    &= \int_{0}^{\infty} -e^{-sx} \sin(ax) \,dx \\
    &= -\int_{0}^{\infty} e^{-sx} \sin(ax) \,dx \\
    \intertext{Using the complex exponential definition of $\sin$}
    &= -\int_{0}^{\infty} e^{-sx} \frac{e^{iax} - e^{-iax}}{2i} \,dx \\
    &= - \frac{1}{2i} \int_{0}^{\infty} e^{-sx}(e^{iax} - e^{-iax})\,dx \\
    &= - \frac{1}{2i} \int_{0}^{\infty} e^{-(s - ia)x} - e^{-(s + ia)x}\,dx \\
    \intertext{Evaluating the improper integral}
    &= - \frac{1}{2i} \left(\frac{-1}{s - ia} e^{-(s - ia)x} - \frac{-1}{s + ia}  e^{-(s + ia)x}\right)\Bigg|_{0}^{x \to \infty} \\
    &= - \frac{1}{2i} \left(0 - \left(\frac{-1}{s - ia} - \frac{-1}{s + ia}\right)\right) \\
    &= - \frac{1}{2i} \left(0 - \left(\frac{-1}{s - ia} - \frac{-1}{s + ia}\right)\right) \\
    &= - \frac{1}{2i} \left(\frac{1}{s -ia} - \frac{1}{s + ia}\right) \\
    &= - \frac{1}{2i} \left(\frac{s + ia}{(s - ia)(s + ia)} - \frac{s - ia}{(s + ia)(s - ia)}\right) \\
    &= - \frac{1}{2i} \left(\frac{s + ia}{s^2 + a^2} - \frac{s - ia}{s^2 + a}\right) \\
    &= - \frac{1}{2i} \left(\frac{2ia}{s^2 + a^2}\right) \\
    I'(s) &= - \frac{a}{s^2 + a^2} \\
    \intertext{Taking the antiderivative of both sides with respect to $s$ to recover the origianal function $I(s)$} 
    \int I'(s) \,ds &= \int - \frac{a}{s^2 + a^2} \,ds  \\
    I(s) &= -\arctan\left(\frac{s}{a}\right) + C \\ 
    \intertext{Since we know the value of the limit $s \to \infty$ of $I(s)$ to be 0, we can take the limit of both sides to find the value of the constant of integration.}
    \lim_{s \to \infty} I(s) &= \lim_{s \to \infty} \left(-\arctan\left(\frac{s}{a}\right) + C\right) \\
    0 &= -\lim_{s \to \infty} \left(\arctan\left(\frac{s}{a}\right)\right) + C \\
    0 &= -\frac{\pi}{2} + C \\ 
    C &= \frac{\pi}{2} \\ 
    I(s) &= -\arctan\left(\frac{s}{a}\right) + \frac{\pi}{2} \\
    \intertext{Recalling how we defined $I$ and $I(s)$, and substituting}
    I &= 2 \cdot I(s = 0) \\
    &= 2 \cdot \left(-\arctan\left(\frac{0}{a}\right) + \frac{\pi}{2}\right) \\
    &= 2 \cdot \left(0 + \frac{\pi}{2}\right) \\ 
    &= \pi \\ \\
    & \boxed{\int_{-\infty}^{\infty} \frac{\sin(ax)}{x} \,dx = \pi}
\end{align*}

\end{document}
